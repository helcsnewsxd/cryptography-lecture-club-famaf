\documentclass{beamer}
\usepackage[utf8]{inputenc}
\usepackage[spanish]{babel}
\usepackage{graphicx}
\usepackage{hyperref}
\usepackage{amsmath}
\usepackage{amsthm}
\usepackage{amssymb}
\usepackage{subcaption}

\usetheme{Madrid}

% Draw
\usepackage{tikz}
\usetikzlibrary{positioning, arrows.meta, shapes.geometric, calc}
\tikzset{
  >={Stealth[length=3mm]},
  box/.style={draw, rounded corners, fill=blue!10, inner sep=6pt, text width=3cm, align=center, minimum height=3cm},
  data/.style={draw, ellipse, fill=green!10, inner sep=4pt, align=center},
}


% Code environment 
\usepackage{minted}
\usepackage{caption}

\setminted{
  mathescape,
  linenos,
  numbersep=3pt,
  frame=lines,
  framesep=1mm,
  baselinestretch=1,
  fontsize=\footnotesize,
  breaklines=true,
  xleftmargin=1em,
  xrightmargin=0em,
}

\newcommand{\codefile}[4][python]{%
  \captionsetup[listing]{name=Código}%
  \captionof{listing}{#2}%
  \label{#3}%
  \inputminted{#1}{#4}%
  }

% Pseudocode environment
\usepackage[ruled,vlined]{algorithm2e}
\renewcommand{\algorithmcfname}{Algoritmo}
\renewcommand{\algorithmautorefname}{Algoritmo}
\renewcommand{\listalgorithmcfname}{Índice de algoritmos}

% Environments
\usepackage{aliascnt}
\setbeamertemplate{theorems}[numbered]
\setbeamertemplate{theorem}[ams style]

\deftranslation[to=spanish]{Theorem}{Teorema}
\deftranslation[to=spanish]{Corollary}{Corolario}
\deftranslation[to=spanish]{Lemma}{Lema}
\deftranslation[to=spanish]{Proof}{Demostración}
\deftranslation[to=spanish]{Definition}{Definición}
\deftranslation[to=spanish]{Example}{Ejemplo}

% Commands 
\usepackage{xparse}
\usepackage[T1]{fontenc}
\usepackage{newtxmath}
\DeclareMathAlphabet{\mathpzc}{T1}{pzc}{m}{it}

\newcommand{\N}{\mathbb{N}}
\NewDocumentCommand{\R}{g g}{
  \IfNoValueTF{#1}
    {\mathbb{R}}
    {
    \IfNoValueTF{#2}
      {\mathbb{R}^{#1}}
      {\mathbb{R}^{#1 \times #2}}
    }
}
\newcommand{\E}{\mathcal{E}}
\newcommand{\K}{\mathcal{K}}
\newcommand{\M}{\mathcal{M}}
\newcommand{\C}{\mathcal{C}}
\newcommand{\A}{\mathcal{A}}
\newcommand{\B}{\mathcal{B}}
\newcommand{\X}{\mathcal{X}}
\newcommand{\Y}{\mathcal{Y}}
\newcommand{\nonce}{{\scriptstyle\mathpzc{N}}}
\newcommand{\Nonce}{\mathpzc{N}}
